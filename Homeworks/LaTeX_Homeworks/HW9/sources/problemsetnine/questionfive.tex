\section{Number of Subintervals by Trapezoidal rule}
	\begin{align}
		&\int_{0}^{2} \sqrt[3]{x+1} \ dx& \\
		\text{tol} &= 10^{-6}& \\
		f(x) &= \sqrt[3]{x+1}& \\
		f'(x) &= \frac{1}{3\sqrt[3]{x+1}} \\
		f''(x) &= -\frac{1}{9(x+1)^{4/3}}& \\
		\max_{0 \leq x \leq 2} |f''(x)| &= |f''(0)| = \left| -\frac{1}{9(0+1)^{4/3}} \right| = \frac{1}{9}&
	\end{align}
	\begin{align}
		\frac{h^{2}}{12}(b-a)f''(c) &\leq \text{tol}& \\
		\frac{h^{2}}{12}(2)\left(\frac{1}{9}\right) &\leq 10^{-6}& \notag \\
		h^{2} &\leq \frac{108*10^{-6}}{2}& \notag \\
		h &\leq \sqrt{5.9*10^{-5}}& \notag \\
		h &\leq 0.00768115& \\
		\frac{b-a}{n} &= h& \\
		n &= \frac{b-a}{h}& \\
		n &\geq \frac{2}{0.00768115}& \notag \\
		n &\geq 260.37768& \notag \\
		\lceil n \rceil &= 261&
	\end{align}
