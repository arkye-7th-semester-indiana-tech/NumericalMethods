\section{Fixed-Points}

	Fixed Point $(p)$ is when the below function is satisfied:

	\begin{equation}
		f(p) = p
	\label{eq:fixedpoint}
	\end{equation}

	\subsection{Question (a)}
		
		Equation:

		\begin{equation}
			f(x) = xln(x+1)
		\label{eq:qfivea}
		\end{equation}

		By simple deduction we discover the only fixed point of \cref{eq:qfivea}:

		\begin{align}
			f(0) &= 0*ln(0+1)& \notag \\
			&= 0*0 = 0& \notag \\
			p &= 0&
			\label{eq:qfiveaded}
		\end{align}

	\subsection{Question (b)}
		
		Equation:

		\begin{equation}
			f(x) = \sqrt{x+1}
		\label{eq:qfiveb}
		\end{equation}

		Testing \cref{eq:qfiveb} with $x = 1$ and $x = 2$:

		\begin{equation}
			f(1) = \sqrt{1+1} = \sqrt{2} > 1
		\label{eq:qfivebone}
		\end{equation}

		\begin{equation}
			f(2) = \sqrt{2+1} = \sqrt{3} < 2
		\label{eq:qfivebtwo}
		\end{equation}

		By deduction, we can affirm the \cref{eq:qfiveb} do not has any fixed point. After multiple attempts to identify a more closely approximation of a possible $p$ using GNU Octave, the value of $x = 1.61803398874989$ is a approximated fixed point of \cref{eq:qfiveb} by 14 significant digits.