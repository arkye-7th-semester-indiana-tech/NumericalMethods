\section{Unique Fixed-Point in an Interval}
	
	\begin{align}
		g(x) &= \frac{1}{5}(x+1)^{\frac{3}{2}}&,[0,1]
	\label{eq:42}
	\end{align}

	\subsection{Question (a)}

		First, it is necessary to check if $g(x)$ is an element of $[a,b]$ for all $x$ in $[a,b]$. Starting with $x = a = 0$:
		
		\begin{align}
			g(x) &= \frac{1}{5}(x+1)^{\frac{3}{2}}& \notag \\
			g(0) &= \frac{1}{5}(0+1)^{\frac{3}{2}}& \notag \\
			&= \frac{1}{5} = 0.2&
		\label{eq:42a0x}
		\end{align}

		Since $0.2$ is an element of $[a,b]$, the next step is to check with $x = b = 1$:

		\begin{align}
			g(x) &= \frac{1}{5}(x+1)^{\frac{3}{2}}& \notag \\
			g(1) &= \frac{1}{5}(1+1)^{\frac{3}{2}}& \notag \\
			&= \frac{2^{\frac{3}{2}}}{5}& \notag \\
			&= 0.56569&
		\label{eq:42a1x}
		\end{align}

		Since $0.56569$ is an element of $[a,b]$, it is correct to assume the \cref{eq:42} can have at least one fixed-point on the indicated interval if it is monotone. Derivating \cref{eq:42}:

		\begin{align}
			g(x) &= \frac{1}{5}(x+1)^{\frac{3}{2}}& \notag \\
			g'(x) &= \frac{1}{5}*\frac{3}{2}*\sqrt{x+1}& \notag \\
			&= \frac{3*\sqrt{x+1}}{10}&
		\label{eq:42derivate}
		\end{align}

		Looking at the \cref{eq:42derivate}, it is correct to affirm that the \cref{eq:42} is monotone in $[a,b]$. The \cref{eq:42} have at least one fixed-point.

		To prove the uniqueness it is necessary to guarantee that $|g'(x)| < 1$ for all $x$ in $[a,b]$. Starting with $x = a = 0$:

		\begin{align}
			g'(x) &= \frac{3*\sqrt{x+1}}{10}& \notag \\
			g'(0) &= \frac{3*\sqrt{0+1}}{10}& \notag \\
			&= \frac{3*\sqrt{1}}{10}& \notag \\
			&= 0.3&
		\label{eq:42derivate0x}
		\end{align}

		Since $0.3 < 1$, the next step is to check $g'(x)$ with $x = b = 1$:

		\begin{align}
			g'(x) &= \frac{3*\sqrt{x+1}}{10}& \notag \\
			g'(1) &= \frac{3*\sqrt{1+1}}{10}& \notag \\
			&= \frac{3*\sqrt{2}}{10}& \notag \\
			&= 0.42426&
		\label{eq:42derivate1x}
		\end{align}

		Since $0.42426 < 1$, and $0.4246$ is the maximum value in $[a,b]$, we can affirm that \cref{eq:42} have an unique fixed-point.

	\subsection{Question (b)}
	
		\begin{align}
			x_{n+1} &= \frac{1}{5}(x_{n}+1)^{\frac{3}{2}}&
		\label{eq:42fixed}
		\end{align}

		\begin{align}
			x_{1} &= \frac{1}{5}(x_{0}+1)^{\frac{3}{2}}& \notag \\
			&= \frac{1}{5} = 0.2&
			\label{eq:42fixed1x}
		\end{align}

		For $k = 0.42426$, $n = 5$, $x_{0} = 0$, and $x_{1} = 0.2$, the error bound is:

		\begin{align}
			| E | &\le \frac{(0.42426)^{5}}{1-0.42426}|0 - 0.2|& \notag \\
			&\le \frac{(0.42426)^{5}}{1-0.42426}|0 - 0.2|& \notag \\
			&\le \frac{0.013745*0.2}{0.57574}& \notag \\
			&\le \frac{0.0027491}{0.57574}& \notag \\
			&\le 0.0047749&
		\label{eq:42error5x}
		\end{align}

	\subsection{Question (c)}

		For $k = \frac{1}{3}$, $E \le 10^{-6}$, $x_{0} = 1$, and $x_{1} = \frac{1}{2}$, the minimum number of iterations necessary is:

		\begin{align}
			\lceil n \rceil &= \frac{\ln{\left(\frac{| 10^{-8} |}{|0 - 0.2|}*\left(1-0.42426\right)\right)}}{\ln{\left(0.42426\right)}}& \notag \\
			&= \frac{\ln{\left(5*10^{-8}*0.57574\right)}}{\ln{\left(0.42426\right)}}& \notag \\
			&= 20.251& \notag \\
			n &= 21&
		\label{eq:42as}
		\end{align}